\documentclass[../OpenAppliedMusicTheory.tex]{subfiles}

\externaldocument{../OpenAppliedMusicTheory}
\externaldocument{checkpoints}

\begin{document}
    
    \sfchapter{Pitches, Scales, and Keys}{3}
    First we learned about note durations and now we will learn about different pitches and how they fit into music. We will start with a bit more of a natural and physical understanding of pitches and move on to organizational structures from there. If you don't really care about the simple physics and math behind this you can skip a bit of this chapter but these ideas can still be useful to know to better understand why certain instruments work certain ways. The physics of music will be covered more in depth in the future, but this is where it begins. %TODO: reference

    \section{How do pitches work in nature?}\label{ch3:pitches-in-nature}

    \section{Notes as pitches}\label{ch3:note-pitches}

    \section{The organization of notes}

\end{document}