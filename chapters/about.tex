\documentclass[../OpenAppliedMusicTheory.tex]{subfiles}

\begin{document}
    
    \sffront{About}

    \rhead{About}

    \section*{Ideologies}
    Music theory can sometimes be a really vague topic. We hear items about theory all the time. \emph{Ooh, that's a perfect cadence} and \emph{there's a picardy third!} But, what do these really mean? I mean yes, there's the definition that we can get on wikipedia or google, but so what? We can give labels all day, but labels are just labels. I can just as easily hear a random melody and say "\emph{Oh wow this is truly exquisite. It's an perfectly implemented inverse sub-transit chain from the C to a B.} It's good to understand how to label certain ideas and concepts--after all, we need some way to communicate with one another--but, the hard part often has more to do with what we can do with this knowledge and how to expand on it. As I mentioned in the probably too long preface of this book, I view music theory in the same way as I view mathematics or physics. Knowledge of the field and its models evolves over time, but old "outdated" ideas can still be applied where appropriate. Additionally, there are differences between laws, theories, hypotheses, etc. This book is based on this ideology. 

    \section*{A different approach}
    With this view of music theory as a science, I hope to use a bit of a different approach. Don't worry though, there won't be heavy math involved :) (or at least not for most of the book). Moving on, the idea is of course to take the good parts of both artistic and scientific (these can be the same thing can't they?) styles of teaching, and combine it with a bit of traditional music pedagogy.
    
    \begin{itemize}
        \item \textbf{Examples.} Examples are king in any type of teaching. When I learn things, I like to be amazed. I like to see something that's really cool, and something that is really cool just by itself. This book loves examples and will use it to help explain just about every single concept that is taught. Sometimes we will introduce an excerpt and ask questions to hopefully bring forth some thought. Sometimes we will introduce an idea in words and then go forth to give examples. In both cases, we hope the examples can really get you thinking about the idea so that you can really understand what it means beyond just the name and how to identify it.
        \item \textbf{Theory (Concepts).} Intuition is a great thing. At the same time, it can be tough as it seems some people have a great intuition for things while other people may have a poor intuition for the same things. From my perspective, though, intuition can be taught. Intuition is something that is built up from experience. This isn't necessarily from direct experience, but experience in general. You might view a physical phenomenon and have an idea of what's happening based on what you've experienced in your life. When something is something people say "isn't intuitive" it's often something that is not experienced in a particular way in practice and in real life. This idea of intuition can be really powerful, but it can be hard to develop. This book tries to develop intuition in order to build practical skills even when you might not be consciously thinking about what you may be doing.
        \item \textbf{Practice.} TODO:
    \end{itemize}

    Now at this point you might be looking at the bullet points and wondering how this is a different approach at all. And you'd be right, because I thought it was different until I reread this. I'm going to keep it here, though, because I still think there is some truth in it. %This is kind of awkward so think about removing it

    \section*{Coverage}

    \section*{What does it mean to understand what something means "beyond just the label"?}

    \section*{How to get the most out of this book}

\end{document}