\documentclass[../OpenAppliedMusicTheory.tex]{subfiles}

\begin{document}
    
    \sffront{About}

    \section*{Ideologies}
    Music theory can sometimes be a really vague topic. We hear items about theory all the time. \emph{Ooh, that's a perfect cadence} and \emph{there's a picardy third!} But, what do these really mean? I mean yes, there's the definition that we can get on wikipedia or google, but so what? We can give labels all day, but labels are just labels. I can just as easily hear a random melody and say "\emph{Oh wow this is truly exquisite. It's an perfectly implemented inverse sub-transit chain from the C to a B.} It's good to understand how to label certain ideas and concepts--after all, we need some way to communicate with one another--but, the hard part often has more to do with what we can do with this knowledge and how to expand on it. As I mentioned in the  preface of this book, I view music theory in the same way as I view mathematics or physics. Knowledge of the field and its models evolves over time, but old "outdated" ideas can still be applied where appropriate. Also, there are differences between laws, theories, hypotheses, etc. This book is based on this ideology. 

    \section*{A different approach}
    With this view of music theory as a science, I hope to use a bit of a different approach. Don't worry though, there won't be heavy math involved :) (or at least not for most of the book). Moving on, the idea is of course to take the good parts of both artistic and scientific (these can be the same thing can't they?) styles of teaching, and combine it with a bit of traditional music pedagogy. To be honest, I'm unsure whether or not this really is a different approach. From my narrow experience with music education (I'm a computer science major after all!), it is a bit of a different approach. In any case, my hope is that this approach can help make music theory accessible for people of all disciplines who may not be super serious about music in academia. In any case, there are three main patterns you will see throughout the book:
    
    \begin{itemize}
        \item[-] \textbf{Examples.} Examples are king in any type of teaching. When I learn things, I like to be amazed. I like to see something that's really cool, and something that is really cool just by itself. This book loves examples and will use it to help explain just about every single concept that is taught. Sometimes we will introduce an excerpt and ask questions to hopefully bring forth some thought. Sometimes we will introduce an idea in words and then go forth to give examples. In both cases, we hope the examples can really get you thinking about the idea so that you can really understand what it means beyond just the name and how to identify it.
        \item[-] \textbf{Theory (Concepts).} Intuition is a great thing. At the same time, it can be tough as it seems some people have a great intuition for things while other people may have a poor intuition for the same things. From my perspective, though, intuition can be taught. Intuition is something that is built up from experience. This isn't necessarily from direct experience, but experience in general. You might view a physical phenomenon and have an idea of what's happening based on what you've experienced in your life. When something is something people say "isn't intuitive" it's often something that is not experienced in a particular way in practice and in real life. This idea of intuition can be really powerful, but it can be hard to develop. This book tries to develop intuition in order to build practical skills even when you might not be consciously thinking about what you may be doing.
        \item[-] \textbf{Practice.} For some people, intuition comes easy. Notes and chords, rhythm, melody--they might just pop up not unlike how you might read a book or generate your speech. It's often easiest to learn a language quickly by moving to a place that speaks that language, or at least by having constant exposure to the language whether you understand everything or not. After maybe even as short as a month, you might be able to hold your own conversation with natives if you put effort into it! The goal of this book isn't to be like a workbook you'd find in your introductory Music Theory class, but to set up the foundations for you to grow efficiently even after you've forgotten about the book and left it in the dust. We want to give easy opportunities to practice, as well as some tools to help you practice theory in (almost) whatever way you want to. After all, there is over a thousand years of music to learn from! :) 
    \end{itemize}

    Intuition is a really nice thing to have, and that's why this book is so focused on it. While I'm a bit of a theory nut, I know a lot of you performers and (sorry!) wannabe pop stars will be a bit bored spending an entire week reading a single piece of music. Music should be fun! When you have to think about every single thing you're doing to write, improvise, perform, etc. it can be a bit draining. That's probably why some people (incorrectly) say that music theory can limit your creativity. By now, I'd think you know I strongly disagree with that idea, but I see why! With a developed intuition for music theory, though, you won't have to always think super hard every time you look at some music. Ideally, you'd unconsciously have an implied "understanding" of what was happening so that you can make something better out of it than if you didn't have that intuition.

    \section*{Coverage}
    This book is an applied music theory book, and as such, will cover aspects of music theory that are useful for reading music, performing music, and writing music. As of this writing, I have yet to write any actual part of the book and so it is difficult to say what will and what won't be covered. In general, though, I hope to discuss both introductory, intermediate, and advanced content. Anything that could be a help to a musician can show up here :)

    This book is a living book that will always be changing. As an open source book, I invite you guys to contribute if you have anything you'd like to add or change. This process will be explained in more detail in the future when it is more complete.  

    \section*{Who is this book for?}
    Anyone with an interest in music can use this book. That being said, basic topics may not be covered as in depth as other sources since there are many great educational materials already out there. The intermediate and advanced material is where the focus of this book is, but keep in mind that it will take time for the material to accrue. If you're currently studying music, this book may be a good supplement that will hopefully give some good alternate perspectives or tips and tricks.

    As for instruments, there is no primary instrument that will necessarily be referred to. Different instruments may pop up in the book to help illustrate an idea, but there is no assumption that the readers will all know any specific instrument.

    \section*{How to get the most out of this book}
    As I've mentioned before, this book follows an approach that hopes to be accessible for people from all disciplines, whether it is music-related or not. This book is meant to be easy to read, but also hold a bit denser content for use as supplementary material. This is not meant to be reference material per se. While it will contain summaries and cheat sheet-esque sections, there may be small details that are not covered completely. I think it is important to note caveats and details, but as an \emph{applied} music theory book, not all details are completely relevant in practice. 

    As for reading the book, you can feel free to read it like a novel. The text should generally be concise, but not in a way that makes it difficult to read or understand. Remember, this book is not meant to be reference material. If you would like, read the book from start to finish. If there is material that you already understand well, feel free to skip it. The book will mention when other chapters might be important or helpful, but the chapters are designed to flow one after another, as well as be able to stand alone.

    %TODO: Consider talking about the importance of examples and practice, because I didn't really explain how to get the most out of the book...

\end{document}