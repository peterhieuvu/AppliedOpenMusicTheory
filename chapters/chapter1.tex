\documentclass[../OpenAppliedMusicTheory.tex]{subfiles}

\externaldocument{../OpenAppliedMusicTheory}
\externaldocument{chapter0}

\begin{document}
    
    \sfchapter{The basics}{1}

    If you've never played a musical instrument and have never been involved in music, I'm not sure how you found this book, but we will gloss over the basics. I will recommend https://www.musictheory.net/ as a better resource for introductory theory. We are not affiliated in any way, but it may be a bit better than just reading through this. In practice, there might be a bit of overlap between that website and this book for the first few chapters. The approaches are different, but in any case, I encourage you to checkout different resources. Different things work for different people. If you decide this book doesn't work well for you, I won't be offended :) But if you don't mind, it would be nice to know what didn't work and maybe how we could improve the book!

    In any case, this chapter will be largely labeling. I mentioned that labeling isn't really the cool part about theory nor is it necessarily the hardest part (not that it's always trivial, but that's for later). Still, it's important for communication and that's why we label things in general. It's good to use words as a way to communicate most effectively and efficiently instead of using buzz words as a way to build your ego because you think you know something other people don't. Love music! Words are a tool, not a direct cure for insecurities you might have. :)

    \section{Notes, rests and the staff}\label{ch.1:notation}
        \paragraph{Picture of some notes here}
        Even if you have never played a musical instrument in your life, you probably have seen notes somewhere. Here, we will give some labels to the different parts of the notes. All of these are notes, but you'll notice that they look different. This is because we draw notes differently depending on how long we want them to be played. 

        From left to right, the notes decrease in duration by a factor of two. If we look at the names then we can kind of think of them as fraction values and compare the relative magnitudes of these fractions. A whole note (\musWhole) would be like $1$, and a half note (\musHalf) would be like $\frac{1}{2}$. Therefore, a half note is half the duration of a whole note. Similarly, a quarter note (\musQuarter) is like $\frac{1}{4}$ and a sixteenth note (\musSixteenth) is like $\frac{1}{16}$. In this case $\frac{1}{16}$ is $4$ times smaller than $\frac{1}{4}$ and so a sixteenth note is four times shorter than a quarter note. Another way to think of this is that there are four sixteenth notes in a quarter note.

        Another way to think of it is visually. Each thing that is added on halves the duration of the note. This is most obvious with quarter notes and shorter. Each time you add a flag, the note halves in duration. This idea is true for shorter durations too. If you take a whole note and add a beam, it becomes a half note which is half the duration. If you take the half note and fill in the circle, it becomes a quarter note which is, again, half the duration.

        %TODO: figure out why the sixteenth note is like this in musicology package

        Now let's put these notes somewhere familiar...
        \paragraph{Picture of notes on the staff with some clefs}
        Explain this.

    \section{TODO:}

\end{document}