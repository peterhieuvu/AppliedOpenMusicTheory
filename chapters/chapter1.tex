\documentclass[../OpenAppliedMusicTheory.tex]{subfiles}

\externaldocument{../OpenAppliedMusicTheory}
\externaldocument{chapter0}

\begin{document}
    
    \sfchapter{Notation}{1} %TODO: consider calling this notation instead

    If you've never played a musical instrument and have never been involved in music, I'm not sure how you found this book, but we will gloss over the basics. I will recommend https://www.musictheory.net/ as a better resource for introductory theory. We are not affiliated in any way, but it may be a bit better than just reading through this. In practice, there might be a bit of overlap between that website and this book for the first few chapters. The approaches are different, but in any case, I encourage you to checkout different resources. Different things work for different people. If you decide this book doesn't work well for you, I won't be offended :) But if you don't mind, it would be nice to know what didn't work and maybe how we could improve the book!

    In any case, this chapter will be largely labeling (hence the name "notation"). I mentioned that labeling isn't really the cool part about theory nor is it necessarily the hardest part (not that it's always trivial, but that's for later). Still, it's important for communication and that's why we label things in general. It's good to use words as a way to communicate most effectively and efficiently instead of using buzz words as a way to build your ego because you think you know something other people don't. Love music! Words are a tool, not a direct cure for insecurities you might have. :)

    \section{Notes}\label{ch1:notes}
        \paragraph{Picture of some notes here}

        Even if you have never played a musical instrument in your life, you probably have seen \textbf{notes} somewhere. Here, we will give some labels to the different parts of the notes. All of these are notes, but you'll notice that they look different. This is because we draw notes differently depending on how long we want them to be played. 

        From left to right, the notes decrease in duration by a factor of two. If we look at the names then we can kind of think of them as fraction values and compare the relative magnitudes of these fractions. A whole note (\musWhole) would be like $1$, and a half note (\musHalf) would be like $\frac{1}{2}$. Therefore, a half note is half the duration of a whole note. Similarly, a quarter note (\musQuarter) is like $\frac{1}{4}$ and a sixteenth note (\musSixteenth) is like $\frac{1}{16}$. In this case $\frac{1}{16}$ is $4$ times smaller than $\frac{1}{4}$ and so a sixteenth note is four times shorter than a quarter note. Another way to think of this is that there are four sixteenth notes in a quarter note.

        Another way to think of it is visually. Each thing that is added on halves the duration of the note. This is most obvious with quarter notes and shorter. Each time you add a flag, the note halves in duration. This idea is true for shorter durations too. If you take a whole note and add a beam, it becomes a half note which is half the duration. If you take the half note and fill in the circle, it becomes a quarter note which is, again, half the duration.

        %TODO: figure out why the sixteenth note is like this in musicology package, also maybe use subsections

        Now let's put these notes somewhere familiar...

    \section{The staff, clefs, and ledger lines}\label{ch1:staff}
        \subsection{The staff}
        \paragraph{Picture of notes on the staff with some clefs}
        
        These 5 lines are called the \textbf{staff} and is where the notes sit. While the type of note determines the duration, the notes' placement on the staff determines the pitch. Higher up on the staff means a higher pitch while lower on the staff means a lower pitch. These notes can be placed on either a line or on a space which is in between two lines. We'll talk about what these mean later.

        \subsection{Clefs}
        You'll also notice some strange looking symbols here on the staff. Those symbols are called clefs and they give reference to which lines or spaces represent which pitches. These pitches are represented by letters between A and G. These letters form a ring, and so if you reach G and go one note further, you'll get back to A. This might seem weird at first, but you can think about a clock and how that works. When you reach 12 o'clock and go one further, you go back to 1 o'clock even though it may be the 13th hour of the day. Each line or space represents one note. If a given line is the note "D", the space above is the note "E" and the space below is the note "C".

        \paragraph{clock figure}

        These clefs, then, determine what letter or note each line and space represent. The clef on the left is called the \textbf{treble clef}. The treble clef is also called the "G clef" because the line it circles (the second line from the bottom) is the note G. In this case, the lines, from bottom to top, are EGBDF. The spaces, consequently, are FACE. Understanding this will require some memorization at first, but eventually should be automatic and intuitive. Remember that the notes are in order. In the worst case, remember one single note and count from there.

        The clef on the right is called the \textbf{bass clef}. The bass clef is also known as the "F clef" because the two dots surround the note F (second line from the top). I'll leave the sequence of lines and spaces for your exercise. 

        \subsection{Ledger lines and the Grand Staff}
        \paragraph{figure of ledger lines}

        Sometimes, we will have notes that go above the range of the staff, either above or below. In those cases, we will use something called ledger lines. They are lines above or below the staff that we place notes on. Think of it like extending the number of lines on the staff above and below. You can keep counting just like before. In the example, then, the note on the ledger line below the staff is C. Since the staff has a treble clef on it, we know that the second line is a G. The bottom line is then E and if we count two more (one space and then one line), we will get to C. 

        If you look at the ledger line above the staff on a bass clef, you'll notice that note is also a C. This is actually the same note as the C below the staff on a treble clef. We can put these two staves together. This is called a grand staff and is often used for instruments like piano which have wide ranges and can play multiple notes at once. The C note that lies in between the staves is called \textbf{middle C}. If we throw a curly brace on the left side of the two staves, we can formally connect then and notate a grand staff.

    \section{Basic Markings (Articulations and Accidentals)}\label{ch1:markings}
        \paragraph{picture of accidentals and articulations}

        In these pictures, you'll now notice some additional markings near the notes. These markings modify some attribute about the notes. 
        
        \subsection{Dotted Notes (Augmentation Dots)}
        The first one on the left is actually not an accidental nor an articulation. It might actually be better fit above in the notes section, but is here with the other disconnected markings for now. It's simply called a dot. Just like it looks. The dot lengthens whatever it is after by a factor of $\frac{1}{2}$. Remember how there are two quarter notes in the space of a half note? %TODO: checkpoint here to test how to fill a half note with something dotted like that
        In this case, there would be a dotted quarter note and a eighth note in the space of a half note. There are two eighth notes in the space of a quarter note and so if a dotted quarter note is 1.5 times the duration, it should last the equivalent duration of 3 eighth notes. There are two quarter notes in a half note which is 4 eighth notes and so that would leave us with $4-3=1$ eighth note in addition to the dotted quarter note to fill the space of a half note.

        Remember how I said that dots lengthen \emph{whatever} they are after by a factor of $\frac{1}{2}$? I deliberately used "whatever" instead of "note" because there can be notes with $2, 3, 4, \dots, \infty$ dots. Of course you won't really see more than 2 or maybe 3 in real world cases. In any case, though, they should be easy to understand if you already understand regular dots. They just multiply the duration of the thing before the dot by 1.5. 
        
        \paragraph{figure of subdivision} %TODO: also maybe introduce subdivision somewhere here

        If we have a double dotted half note, well let's just break it down. The first dot adds half of the half note. Half of a half note is a quarter note. Now for the second dot we add another half of that. Half of a quarter note is an eighth note. In total, we have one half note plus a quarter note plus an eighth note. An additional dot would add a sixteenth note and so on. \emph{Fun fact: if there were infinite dots after a given note, the resultant note would have a duration equal to precisely twice the original note.}

        \subsection{Articulations}

        \paragraph{picture of articulation part of the above picture}
        These are different markings that can affect different things about the notes and how they are played. There are tons of different markings and it's good to know the common ones. They are basically just definitions, though, so we won't cover this in great detail. Someday this book may include a cheat sheet of as many articulations as we know and what each of them mean. %TODO:
        Sometimes, they also can have different meanings depending on the instrument. As a result, this is maybe a bit more suited for a composition book or performance books.

        \subsection{Accidentals}
        \paragraph{picture of some notes with accidentals}
        These markings are \textbf{accidentals}. To understand this idea, let's first take a look at a piano keyboard.

        \paragraph{picture of keyboard}
        Here, you'll see that there are white keys and black keys. Can you find any patterns? There should be a repeating pattern every 12 keys made up of groups of 2 black keys and groups of 3 black keys. Counting the white keys, there are 7 notes before you complete a cycle of the pattern. Remember the letters we assigned to these notes? A-G? That's also 7 letters. These letters match up with the white keys on a keyboard with C being the first white note to the left and adjacent to any group of two black keys. 

        \paragraph{picture of keyboard and some labels}
        So now, we need a way to notate those black keys that are in between some white keys. We do this using accidentals. There are four accidentals in figure %TODO figure ref
        PLACEHOLDER. The first two are sharps and the second two are flats. The first sharp is just a standard sharp and raises the pitch by what is called a \textbf{semi-tone}. A semi-tone is the distance between two adjacent keys on the keyboard. Looking at the keyboard this can mean white key to black key, black key to white key, and white key to white key. The second sharp is a double sharp. This is just the equivalent of two sharps which raises the pitch by a total of 2 semi-tones or 1 \textbf{whole tone}.

        Flats do the opposite. The third symbol lowers the pitch by one semi-tone and the fourth symbol is a double flat that lowers the pitch by 2 semi-tones. Here is how it looks on a keyboard:

        \paragraph{picture of keyboard with notes labeled}
        You'll notice that every key has multiple names like C\sh/D\fl. This should make sense. We have learned so far that naked (without accidentals) or \textbf{natural} notes A, B, C, \dots, G are just the plain white keys. The first black key in a group of two is immediately to the right of what we learned to be C and so that note is C\sh. At the same time, we know that that key is to the left of D and so it must be D\fl. This is called \textbf{enharmonic equivalence}: two different note spellings that sound the same.\footnote{We'll find later that this isn't always 100\% true. %TODO: Reference when this part is written
        With different temperaments enharmonic equivalences can sound different, but this doesn't really apply as much anymore with modern standards }
        This also means that D note is C\musDoubleSharp\ and E\musDoubleFlat, but we left that off the diagram because it would be too messy.

        One thing to note, however, is the adjacent white keys. E and F are two white keys that are directly next to each other. That means that E\sh does not actually land on a black key like most other accidentals. This isn't necessarily confusing, but can be a common mistake to make since E\sh\ and similar notes don't show up in music as much as other notes.


\end{document}