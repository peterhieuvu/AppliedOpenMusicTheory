\documentclass[../OpenAppliedMusicTheory.tex]{subfiles}

\externaldocument{../OpenAppliedMusicTheory}
\externaldocument{chapter0}

\begin{document}
    
    \sfchapter{Notation}{1} %TODO: consider calling this notation instead

    If you've never played a musical instrument and have never been involved in music, I'm not sure how you found this book, but we will gloss over the basics. I will recommend https://www.musictheory.net/ as a better resource for introductory theory. We are not affiliated in any way, but it may be a bit better than just reading through this. In practice, there might be a bit of overlap between that website and this book for the first few chapters. The approaches are different, but in any case, I encourage you to checkout different resources. Different things work for different people. If you decide this book doesn't work well for you, I won't be offended :) But if you don't mind, it would be nice to know what didn't work and maybe how we could improve the book!

    In any case, this chapter will be largely labeling (hence the name "notation"). I mentioned that labeling isn't really the cool part about theory nor is it necessarily the hardest part (not that it's always trivial, but that's for later). Still, it's important for communication and that's why we label things in general. It's good to use words as a way to communicate most effectively and efficiently instead of using buzz words as a way to build your ego because you think you know something other people don't. Love music! Words are a tool, not a direct cure for insecurities you might have. :)

    \section{Notes}\label{ch1:notes}
        \paragraph{Picture of some notes here}

        Even if you have never played a musical instrument in your life, you probably have seen \textbf{notes} somewhere. Here, we will give some labels to the different parts of the notes. All of these are notes, but you'll notice that they look different. This is because we draw notes differently depending on how long we want them to be played. 

        From left to right, the notes decrease in duration by a factor of two. If we look at the names then we can kind of think of them as fraction values and compare the relative magnitudes of these fractions. A whole note (\musWhole) would be like $1$, and a half note (\musHalf) would be like $\frac{1}{2}$. Therefore, a half note is half the duration of a whole note. Similarly, a quarter note (\musQuarter) is like $\frac{1}{4}$ and a sixteenth note (\musSixteenth) is like $\frac{1}{16}$. In this case $\frac{1}{16}$ is $4$ times smaller than $\frac{1}{4}$ and so a sixteenth note is four times shorter than a quarter note. Another way to think of this is that there are four sixteenth notes in a quarter note.

        Another way to think of it is visually. Each thing that is added on halves the duration of the note. This is most obvious with quarter notes and shorter. Each time you add a flag, the note halves in duration. This idea is true for shorter durations too. If you take a whole note and add a beam, it becomes a half note which is half the duration. If you take the half note and fill in the circle, it becomes a quarter note which is, again, half the duration.

        %TODO: figure out why the sixteenth note is like this in musicology package, also maybe use subsections

        Now let's put these notes somewhere familiar...

    \section{The staff and clefs}\label{ch1.staff}
        \paragraph{Picture of notes on the staff with some clefs}
        
        These 5 lines are called the \textbf{staff} and is where the notes sit. While the type of note determines the duration, the notes' placement on the staff determines the pitch. Higher up on the staff means a higher pitch while lower on the staff means a lower pitch. These notes can be placed on either a line or on a space which is in between two lines. We'll talk about what these mean later.

        You'll also notice some strange looking symbols here on the staff. Those symbols are called clefs and they give reference to which lines or spaces represent which pitches. These pitches are represented by letters between A and G. These letters form a ring, and so if you reach G and go one note further, you'll get back to A. This might seem weird at first, but you can think about a clock and how that works. When you reach 12 o'clock and go one further, you go back to 1 o'clock even though it may be the 13th hour of the day. Each line or space represents one note. If a given line is the note "D", the space above is the note "E" and the space below is the note "C".

        \paragraph{clock figure}

        These clefs, then, determine what letter or note each line and space represent. The clef on the left is called the \textbf{treble clef}. The treble clef is also called the "G clef" because the line it circles (the second line from the bottom) is the note G. In this case, the lines, from bottom to top, are EGBDF. The spaces, consequently, are FACE. Understanding this will require some memorization at first, but eventually should be automatic and intuitive. Remember that the notes are in order. In the worst case, remember one single note and count from there.

        The clef on the right is called the \textbf{bass clef}. The bass clef is also known as the "F clef" because the two dots surround the note F (second line from the top). I'll leave the sequence of lines and spaces for your exercise. 

    \section{Ledger lines and the Grand Staff}\label{ch1.ledgerlines}

        \paragraph{figure of ledger lines}

        Sometimes, we will have notes that go above the range of the staff, either above or below. In those cases, we will use something called ledger lines. They are lines above or below the staff that we place notes on. Think of it like extending the number of lines on the staff above and below. You can keep counting just like before. In the example, then, the note on the ledger line below the staff is C. Since the staff has a treble clef on it, we know that the second line is a G. The bottom line is then E and if we count two more (one space and then one line), we will get to C. 

        If you look at the ledger line above the staff on a bass clef, you'll notice that note is also a C. This is actually the same note as the C below the staff on a treble clef. We can put these two staves together. This is called a grand staff and is often used for instruments like piano which have wide ranges and can play multiple notes at once. If we throw a curly brace on the left side of the two staves, we can formally connect then and notate a grand staff.

    \section{Accidentals and Articulation}

\end{document}