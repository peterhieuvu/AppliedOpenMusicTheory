\documentclass[../OpenAppliedMusicTheory.tex]{subfiles}

\externaldocument{../OpenAppliedMusicTheory}
\externaldocument{chapter0}

\begin{document}
    
    \sfchapter{The basics}{1}

    If you've never played a musical instrument and have never been involved in music, I'm not sure how you found this book, but we will gloss over the basics. I will recommend https://www.musictheory.net/ as a better resource for introductory theory. We are not affiliated in any way, but it may be a bit better than just reading through this. In practice, there might be a bit of overlap between that website and this book for the first few chapters. The approaches are different, but in any case, I encourage you to checkout different resources. Different things work for different people. If you decide this book doesn't work well for you, I won't be offended :) But if you don't mind, it would be nice to know what didn't work and maybe how we could improve the book!

    In any case, this chapter will be largely labeling. I mentioned that labeling isn't really the cool part about theory nor is it necessarily the hardest part (not that it's always trivial, but that's for later). Still, it's important for communication and that's why we label things in general. It's good to use words as a way to communicate most effectively and efficiently instead of using buzz words as a way to build your ego because you think you know something other people don't. Love music! Words are a tool, not a direct cure for insecurities you might have. :)

    \section{Notes, rests and the staff}\label{ch.1:notation}
        \paragraph{Picture of some notes here}
        Explain the parts of the notes and what we call them here.

        Let's put them somewhere familiar...
        \paragraph{Picture of notes on the staff with some clefs}
        Explain this.

    \section{TODO:}

\end{document}