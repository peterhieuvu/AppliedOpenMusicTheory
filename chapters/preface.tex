\documentclass[../OpenAppliedMusicTheory.tex]{subfiles}

\begin{document}
    
    \sffront{Preface}

    \rhead{Preface}

    This preface is to be rewritten once I have finished a release of this book. As of this moment, though, I am a young naive undergraduate student studying Computer Science and Mathematics. Why, then, am I writing a music theory book? I don't have a good reason for that, but here we go...

    Fountain Pens and happy friends. 

    I remember the moment when I took my very first music theory course. I was a high schooler looking to register for some classes at a local community college for a program called Running Start. Calculus was full and so, being a clueless Sophomore, I took "Music Theory/Ear Training 1" as an easy class. Little did I know that I would be sucked into a new world that I never could have known before. Ron Bayer was my theory professor and he really showed me what music theory was capable of. No, it wasn't some silly ruleset that musicians use to shake their heads at others. No, it wasn't a silly set of restrictions that musicians use to make top pop hits. In a way, it can be seen like that, but I should have known better as a STEM nut. I discovered that music theory was theory just like Einstein's Theory of Relativity or the Theory of Heliocentrism developed by Copernicus, Kepler, and Galileo. Based on our observations, we've noticed things about how the world works. As we investigate further we begin to find ways to explain why these observations happen. Over time as we experiment and advance, we modify our understanding of the world. Some explanations we'll find to fail completely. Others we'll find still hold true. Many we will find to be obsolete, and proceed to make enhancements and adjustments. 

    But! Just because one system fails to completely explain a phenomenon does not mean we must scrap this idea completely. Many of us are familiar with the concept of relativity (Galilean Relativity for you physics folks) If you're standing on a cruise ship moving $5m/s$ and walk $.5m/s$ in the same direction as the ship, you'll be moving $5+.5=5.5m/s$ relative to an observer on land! Eventually, however, we found out that this addition identity doesn't hold true at high speeds. A very smart person named Albert Einstein developed a new model of Special Relativity in order to explain the newly observed phenomenon. And notice that many people will never get to the point of even learning special relativity even though Galilean relativity is incorrect. Why? Well because it's close enough and most of us don't care enough. The fact that there is a time dilation for someone driving a car on a freeway compared to a speed limit sensor doesn't matter. Even an engineer won't even necessarily need to account for these effects. And these engineers are professionals! We're entrusting our lives to them yet they use theories that are incorrect?

    Okay I'll admit that my analogy isn't the best, but music theory should be seen the same way. First, we're asking questions and making observations about what is being done. We might notice that \chord{G7} chords often go to \chord{C} chords. From there, we can break down the observation and make hypotheses. 

\end{document}